En la construcción de un algoritmo, existen variadas herramientas que nos permiten dotar de ciertas características específicas, que dependiendo del funcionamiento diseñado y tomando en consideración el ámbito en que se ejecutará, nos convendrán más o menos. Derivado de esto, nos es obligatorio conocer a fondo la naturaleza de la problematica a resolver, y nos es útil conocer para esto, las técnicas que mejor se ajusten a nuestras necesidades y las del algoritmo.

Los algoritmos voraces, o también llamados \textit{Greedy}, son un tipo particular de algoritmos definidos por su manera de organizar los elementos y atacar al problema. En términos generales, estos algoritmos suelen tener un diseño menos complejo pues el horizonte del panorama que considera cuando se realizan decisiones, suele estar acotado localmente, y entre un conjunto de opciones escogerá aquella que ofrece un beneficio mayor inmediato.

Por esta misma característica, es de suma importancia comprender si el problema que se intenta resolver, requisita de alguna solución óptima pues el impacto de no resolver la más óptima es despreciable. Existiendo también problemas donde a partir de la elección de los elementos locales más óptimos nos lleva a una solución global aceptable. Este tipo de algoritmos cumplen su cometido sin la necesidad de realizar un esfuerzo mayor de diseño, que probablemente otra solución pudiera exigir.\\

A partir de su utilidad, se analiza un par de algoritmos conocidos por su eficiencia y capacidad de otorgar una solución óptima mediante la aplicación de este paradigma: El algoritmo para la compresión de texto de Huffman, y el algoritmo para la búsqueda de árboles recubridores mínimos.