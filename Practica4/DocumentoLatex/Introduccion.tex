Desde el surgimiento de los primeros computadores, la creación de algoritmos eficientes requiería de un duro trabajo intelectual para los programadores de ese tiempo, con los avances de la tecnología y la actualización más constante del hardware, las computadoras crecieron su capacidad de almacenamiento y cálculo, permitiendo ejecutar una grama mayor de tipos de algoritmos para cualquier problema, más la búsqueda de diferentes técnicas para la creación algorítmica, dirigió eventualmente a las mentes expertas matemáticas, a dominar las propiedades de recursividad, permitiendo así, el nacimiento de un nuevo tipo de algoritmos, que de principio requieren un análisis experto y la verificación mediante los recursos matemáticos. Es así que la técnica de divide y vencerás tomo notoriedad, pues una gran cantidad de algoritmos eficientes que se utilizan en la actualidad para resolver variedad de problemas se basan en esta.