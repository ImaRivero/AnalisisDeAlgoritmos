Para la creación de un algoritmo existen distintos métodos y enfoques ahora ya bastante bien conocidos y estudiados, tales como los secuenciales, los recursivos, los Greedy o voraces, etc. Y además de los anteriores aquellos que nos encontramos analizando en esta práctica, los que implementan la programación dinámica mediante el uso de memoización, en la mayoria de los casos. Sin embargo, para hacer uso de estos enfoques nuestro problema deberá cumplir con ciertas condiciones, que más adelante serán mencionadas.

Como podemos ver, se cuenta con una amplia gama de técnicas y muy probablemente no sea tan sencillo escoger entre ellas, pero finalmente es tarea de la ingeniería conocer todos estas técnicas y enfoques para finalmente elegir cúal es el que hace mejor se adapta a la problematica que se busca resolver.

Los algoritmos que se analizan y prueban en este documento son algoritmos clásicos bien estudiados por todos aquellos profesionales pertenecientes a las ciencias de la computación, pero esta reputación de clásicos, no se debe solamente a que ejemplifican perfectamente el uso de los métodos con las que son implementadas, pero por que son simplemente útiles para la resolución de problemáticas con las que cotidianamente lidian, tanto profesionales como no expertos en el área donde surge la necesidad de encontrar solución a una situación.

Siendo una de las sucesiones de números más fascinantes que encontramos directamente describiendo fenómenos de la naturaleza con Fibonacci. O el amplio abanico de problemas en los que puede implementarse, mediante variaciones del problema original, la Mochila entera para encontrar la solución a problemas específicos de un área.